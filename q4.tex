\section{Question 4}

\subsection{part a.}
We prove the equivalence of  \[
(\neg P_1 \lor ... \lor \neg P_m \lor Q) = (P_1 \land ... \land P_m \to Q)
\]
Start with
\[
(\neg P_1 \lor ... \lor \neg P_m \lor Q)
\]
Using DeMorgan's theorem, this term is equivalent to: 
\[
\neg(P_1 \land ... \land P_m \land \neg Q)
\]
Using the Associative property:
\[
\neg((P_1 \land ... \land P_m) \land \neg Q)
\]
Using the conditional equivalence: $\neg(P \to Q) = (P \land \neg Q)$
\[
\neg(\neg(P_1 \land ... \land P_m \to Q))
\]
Finally,
\[
(P_1 \land ... \land P_m \to Q)
\]

\subsection{part b.}
We can repeat the above proof by replacing Q with an expression $(Q_1 \lor ... \lor Q_n)$. A literal can be replaced with an expression. 

\[
(\neg P_1 \lor ... \lor \neg P_m \lor \neg (Q_1 \lor ... \lor Q_n))
\]
Using DeMorgan's theorem, this term is equivalent to:
\[
\neg(P_1 \land ... \land P_m \land \neg (Q_1 \lor ... \lor Q_n))
\]
Using the Associative property
\[
\neg((P_1 \land ... \land P_m) \land \neg (Q_1 \lor ... \lor Q_n))
\]
Using the conditional equivalence: $\neg(P \to Q) = (P \land \neg Q)$
\[
\neg(\neg((P_1 \land ... \land P_m) \to (Q_1 \lor ... \lor Q_n)))
\]
Finally,
\[
(P_1 \land ... \land P_m) \to (Q_1 \lor ... \lor Q_n)
\]

\subsection{part c.}
To complete the full resolution and find the resolvent, start with the two clauses:
\[
(l_1 \lor ... \lor l_k \lor m_1 \lor ... \lor m_n)
\]
We can split this expression up using the associative property:
\[
(l_i \lor m_j) \lor (l_1 \lor ... \lor l_{i-1} \lor l_{i+1} \lor ... \lor l_k) \lor (m_1 \lor ... \lor m_{j-1} \lor m_{j+1} \lor ...  \lor m_n)  
\]
Using the Conditional equivalence property, similar to what we derived in part A, this term is equivalent to:
\[
\neg(l_i \lor m_j) \to (l_1 \lor ... \lor l_{i-1} \lor l_{i+1} \lor ... \lor l_k) \lor (m_1 \lor ... \lor m_{j-1} \lor m_{j+1} \lor ...  \lor m_n)  
\]
Because $l_i$ and $m_j$ are complimentary:
$(l_i \lor m_j) = True$
\[
True \to (l_1 \lor ... \lor l_{i-1} \lor l_{i+1} \lor ... \lor l_k) \lor (m_1 \lor ... \lor m_{j-1} \lor m_{j+1} \lor ...  \lor m_n)  
\]
Therefore it is inferred that  
\[
(l_1 \lor ... \lor l_{i-1} \lor l_{i+1} \lor ... \lor l_k \lor m_1 \lor ... \lor m_{j-1} \lor m_{j+1} \lor ...  \lor m_n)
\]