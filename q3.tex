\section{Question 3}
\subsection{part a.}
S = superman is defeated \\
O = facing opponent alone \\
OK = opponent is carrying kryptonite \\
For superman to be defeated, it has to be that he is facing an opponent alone and his opponent is carrying kryptonite. \\
\[
S \iff (O \land OK)  \]
\[
(S \to (O \land OK)) \land ((O \land OK) \to S)                            \text{(applied biconditional elimination)} \]
\[
(S \to (\neg{O} \lor \neg{OK})) \land ((O \land OK) \to S)                 \text{(DeMorgans)}                         
\]
\[
(\neg{S} \lor (\neg{O} \lor \neg{OK})) \land (\neg{(O \land OK)} \lor S)   \text{(applied implication elimination)}  \] 
\[
(\neg{S} \lor \neg{O} \lor \neg{OK})  \land (\neg{O} \lor \neg{OK} \lor S) \text{(Distributivity)                }    
\] 

AK = acquiring kryptonite \\
BC = batman coordinates with lex luther \\
BA = batman acquires kryptonite from lex \\
\noindent
Acquiring kryptonite, however, means that batman has to coordinate with lex luther and
acquire it from him. \\
\[
AK \to (BC \land BA) \]
\[\neg AK \lor (BC \land BA) \]
\[\neg AK \lor \neg BC \lor \neg BA \]

WU = wonder woman is upset \\
WS = wonder woman fights with superman \\
If, however, batman coordinates with lex luther, this upsets wonder woman who will intervene and fight on the side of superman\\
\[
BC \to WU \land WS \]
\[\neg BC \lor WU \lor WS
\]
\subsection{part b.}
Turning the above statements into 3-cnf form, our KB can be represented as follows...\\
\[
(\neg{S} \lor \neg{O} \lor \neg{OK})  \land (\neg{O} \lor \neg{OK} \lor S)  \land (\neg AK \lor \neg BC \lor \neg BA) \land (\neg BC \lor WU \lor WS)
\]
\subsection{part c.}
To show that batman cannot defeat superman, we can show that $(KB \land \alpha)$ is unsatisfiable(The knowledge base entails that superman can be defeated),
which means every assignment of variables does not satisfy the sentence. This is known as proof by contradiction ${\alpha} = S$, or the value of superman
being defeated by batman.
\newpage
Taking the disjunction of KB and $\alpha... $\\
\[
(\neg{S} \lor \neg{O} \lor \neg{OK})  \land (\neg{O} \lor \neg{OK} \lor S)  \land (\neg AK \lor \neg BC \lor \neg BA) \land (\neg BC \lor WU \lor WS) \land S \]
\[(\neg{O} \lor \neg{OK}) \land (S \land \neg{S})  \land (\neg AK \lor \neg BC \lor \neg BA) \land (\neg BC \lor WU \lor WS) \land S 
\]
We can apply the resolution rule and note that since we have $\neg{S}$ and S it will resolve to the empty clause. This shows
that the $(KB \land \alpha)$ is not satisfiable and therefore shows that Batman can in fact not defeat superman.
