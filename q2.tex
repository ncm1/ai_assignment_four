\section{Question 2}

\subsection{part a.}
The set of variables is the combination of the row and column of the grid. \\
X = {[1,1], [1,2], ... , [i,j-1], [i, j]}

The domain is the possible values that each variable can hold: 1 to 9. \\
D = {1, 2, 3, 4, 5, 6, 7, 8, 9}

The constraint is that each row, column, and neighboring box must hold unique values in each of the variables.

i, j = fixed variable
I, J = random variable
C = {(X(1,1), X(1,2), X(1,3), X(1,4), X(1,5), X(1,6), X(1,7), X(1,8), X(1,9)), X(1,1) != X(1,2) != X(1,3) != X(1,4) != X(1,5) != X(1,6) != X(1,7) != X(1,8) != X(1,9)}

\subsection{part b.}
Start state: initial arrangement of values 1-9 on the Sudoku board in random locations and such that the constraints are not broken. No additional values on the board \\
Successor function: add

Goal test: all locations on the board has a value between 1 - 9.

Path cost: If the action leads to a state where one of the constraints are broken, then the path cost should be infinite. Path cost should depend on how many

\subsection{part c.}
Easy problems can be solved by choosing actions in the coordinates that are most constrained, and continue choosing actions this way until the board is filled. The clues are given in such a way that every action the user takes can be deterministic and the user does not have to make any choices.


Difficult problems most likely will require backtracking, because the solution can require actions that aren't intuitive and may require the user to make approximations. The puzzle may reach an equilibrium point where the next action is not obvious, as multiple actions can be taken and still be valid until a much later state.

\subsection{part d.}
NOT DONE
