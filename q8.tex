\section{Question 8}

\subsection{part a.}
Cost of good quality car:
\begin{equation}
\label{eq:c+}
C(q^+(c_1)) = \$ 4000 - (3000) = \$1000.
\end{equation}
Cost of bad quality car:
\begin{equation}
\label{eq:c-}
C(q^-(c_1)) = \$ 4000 - (3000 + 1400) = -\$400. 
\end{equation}
\textit{Assuming that repairs can be made without taking it to the mechanic - perhaps the \$1400 is a separate cost than the \$100 needed to check the quality of the car.} \\
Probability of good quality car:
\begin{equation}
\label{eq:+}
	P(q^+(c_1)) = 0.7
\end{equation}
Probability of bad quality car:
\begin{equation}
\label{eq:-}
	P(q^-(c_1)) = 0.3
\end{equation}

Expected net gain:
\[E(c_1) = C(q^+(c_1))*P(q^+(c_1)) + C(q^-(c_1))*P(q^-(c_1))\]
\begin{equation} 
\label{eq:expected_value} 
\mathbf{E(c_1) = \$580} 
\end{equation}


\subsection{part b.}

\begin{equation}
\label{eq:pass+}
	P(Pass | q^+) = 0.8
\end{equation}
\begin{equation}
\label{eq:pass-}
	P(Pass | q^-) = 0.35
\end{equation}
Using \ref{eq:pass+} and \ref{eq:pass-}:
\begin{equation}
\label{eq:npass+}
	P(\neg Pass | q^+) = 1 - P(Pass | q^+) = 0.2
\end{equation}
\begin{equation}
\label{eq:npass-}
	P(\neg Pass | q^-) = 1 - P(Pass | q^-) = 0.65
\end{equation}
Using \ref{eq:+}, \ref{eq:-}, \ref{eq:pass+}, \ref{eq:pass-}:
\[P(Pass) = P(Pass | q^+) * P(q^+) + P(Pass | q^-) * P(q^-) = 0.665\]
\begin{equation}
\label{eq:pass}
	\mathbf{P(Pass) = 0.665}
\end{equation}
Using \ref{eq:pass}:
\[P(\neg Pass) = 1 - P(Pass) = 0.335  \]
\begin{equation}
\label{eq:npass}
	\mathbf{P(\neg Pass) = 0.335}
\end{equation}

Using \ref{eq:pass+}, \ref{eq:+}, and \ref{eq:pass}:
\begin{equation}
	\mathbf{P(q^+ | Pass) = \frac{P(Pass | q^+) * P(q^+)}{P(Pass)}  = 0.842}
\end{equation}
Using \ref{eq:pass-}, \ref{eq:-}, and \ref{eq:pass}:
\begin{equation}
	\mathbf{P(q^- | Pass) = \frac{P(Pass | q^-) * P(q^-)}{P(Pass)}  = 0.158}
\end{equation}
Using \ref{eq:npass+}, \ref{eq:+}, and \ref{eq:npass}:
\begin{equation}
	\mathbf{P(q^+ | \neg Pass) = \frac{P(\neg Pass | q^+) * P(q^+)}{P(\neg Pass)}  = 0.418}
\end{equation}
Using \ref{eq:npass-}, \ref{eq:-}, and \ref{eq:npass}:
\begin{equation}
	\mathbf{P(q^- | \neg Pass) = \frac{P(\neg Pass | q^-) * P(q^-)}{P(\neg Pass)}  = 0.582}
\end{equation}

\subsection{part c.}
Paying for the test with the mechanic, the new costs are:
\[ C'(q^+(c_1)) = C(q^+(c_1)) - \$100 = \$900 \]
\[ C'(q^-(c_1)) = C(q^-(c_1)) - \$100 = -\$500 \]

Given a pass:
\[E(c_1 | Pass) = C'(q^+(c_1))*P(q^+(c_1)|Pass) + C'(q^-(c_1))*P(q^-(c_1)|Pass)\]
\[\mathbf{E(c_1 | Pass) = \$678.8}\]

Given a failure:
\[E(c_1 |\neg Pass) = C'(q^+(c_1))*P(q^+(c_1)|\neg Pass) + C'(q^-(c_1))*P(q^-(c_1)|\neg Pass)\]
\[\mathbf{E(c_1 |\neg Pass) = \$85.2}\]

Regardless of a pass or a failure, the best decision is the sell the car as there will be a net gain.

\subsection{part d.}
Without the mechanic's test, the expected gain from selling the car will be 
$\mathbf{E(c_1) = \$580} $

With the test, the expected gain is $\mathbf{ \$678.80}$. The value of the optimal information is the difference between the expected gain with the information and the expected gain without the information. The optimal information value is $\mathbf{\$98.80}$.
I should take C1 to the mechanic.

